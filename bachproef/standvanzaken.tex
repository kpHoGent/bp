\chapter{\IfLanguageName{dutch}{Stand van zaken}{State of the art}}%
\label{ch:stand-van-zaken}

% Tip: Begin elk hoofdstuk met een paragraaf inleiding die beschrijft hoe
% dit hoofdstuk past binnen het geheel van de bachelorproef. Geef in het
% bijzonder aan wat de link is met het vorige en volgende hoofdstuk.

% Pas na deze inleidende paragraaf komt de eerste sectiehoofding.

\section{\IfLanguageName{dutch}{Artificiële Intelligentie}{Artificial Intelligence}}%
\label{sec:artificiële-intelligentie}

Artificiële Intelligentie (AI) is een dynamisch en snel evoluerend vakgebied dat zich richt op het ontwikkelen van systemen die taken kunnen uitvoeren die normaal gesproken menselijke intelligentie vereisen, zoals leren, probleemoplossing en besluitvorming \autocite{SharifaniEtAl2023}.

Het doel is om machines te ontwikkelen die autonoom kunnen functioneren in complexe en dynamische omgevingen \autocite{Kouassi2023}.

AI als vakgebied is al meer dan 65 jaar in ontwikkeling en heeft zich inmiddels diep genesteld in ons dagelijks leven. Het speelt immers een curicale rol in sectoren zoals gezondheidszorg, transport, onderwijs en industrie, en wordt gezien als een belangrijke drijfveer voor sociaaleconomische veranderingen en technologische vooruitgang \autocite{JiangEtAl2022}.

Binnen AI zijn Machine Learning (ML) en Deep Learning (DL) twee van de meest revolutionaire technologieën, die de afgelopen jaren aanzienlijke vooruitgang hebben geboekt \autocite{SharifaniEtAl2023}.

\section{\IfLanguageName{dutch}{Machine Learning}{Machine Learning}}%
\label{sec:machine-learning}

ML is momenteel de meest dominante vorm van AI. Het is een methode voor data-analyse die het mogelijk maakt om analytische modellen automatisch te bouwen en te verbeteren. Het stelt computers in staat om te leren van ervaring, zonder expliciet te worden geprogrammeerd \autocite{SharifaniEtAl2023}.

ML omvat de volgende technieken, die gecombineerd kunnen worden om nog krachtigere en veelzijdigere AI-systemen te ontwikkelen \autocite{Kouassi2023}. 

\begin{itemize}
  \item \textbf{Supervised Learning (SL)}: Hierbij wordt een model getraind met behulp van gelabelde gegevens, waarbij zowel de invoer als de gewensteuitvoer bekend zijn. Het model leert in feite door voorbeelden, vergelijkbaar met een leerling die oefent met vragen en de bijbehorende antwoorden. Het doel is om patronen te herkennen die vervolgens kunnen worden gebruikt om voorspellingen te maken voor nieuwe, onbekende gegevens. Voorbeelden van toepassingen zijn het herkennen van spam en het classificeren van afbeeldingen.
  \item \textbf{Unsupervised Learning (UL)}: Deze techniek werkt met ongelabelde data, waarbij het model zelf patronen of structuren moet ontdekken. Dit gebeurt vaak door clustering, waarbij vergelijkbare data automatisch worden gegroepeerd. Een voorbeeld is het segmenteren van klanten op basis van koopgedrag. UL is vooral nuttig wanneer er geen duidelijke labels beschikbaar zijn. 
  \item \textbf{Reinforcement Learning (RL)}: RL draait om een model dat leert door interactie met een omgeving. Het model probeert een strategie (beleid) te ontwikkelen die maximale beloningen oplevert. Dit wordt vaak gebruikt in scenario's zoals robotica, gaming en autonome voertuigen, waarbij het model leert door trial-and-error. 
\end{itemize}


