%%=============================================================================
%% Samenvatting
%%=============================================================================

% TODO: De "abstract" of samenvatting is een kernachtige (~ 1 blz. voor een
% thesis) synthese van het document.
%
% Een goede abstract biedt een kernachtig antwoord op volgende vragen:
%
% 1. Waarover gaat de bachelorproef?
% 2. Waarom heb je er over geschreven?
% 3. Hoe heb je het onderzoek uitgevoerd?
% 4. Wat waren de resultaten? Wat blijkt uit je onderzoek?
% 5. Wat betekenen je resultaten? Wat is de relevantie voor het werkveld?
%
% Daarom bestaat een abstract uit volgende componenten:
%
% - inleiding + kaderen thema
% - probleemstelling
% - (centrale) onderzoeksvraag
% - onderzoeksdoelstelling
% - methodologie
% - resultaten (beperk tot de belangrijkste, relevant voor de onderzoeksvraag)
% - conclusies, aanbevelingen, beperkingen
%
% LET OP! Een samenvatting is GEEN voorwoord!

%%---------- Nederlandse samenvatting -----------------------------------------
%
% TODO: Als je je bachelorproef in het Engels schrijft, moet je eerst een
% Nederlandse samenvatting invoegen. Haal daarvoor onderstaande code uit
% commentaar.
% Wie zijn bachelorproef in het Nederlands schrijft, kan dit negeren, de inhoud
% wordt niet in het document ingevoegd.

\IfLanguageName{english}{%
\selectlanguage{dutch}
\chapter*{Samenvatting}
\selectlanguage{english}
}{}

%%---------- Samenvatting -----------------------------------------------------
% De samenvatting in de hoofdtaal van het document

\chapter*{\IfLanguageName{dutch}{Samenvatting}{Abstract}}

Deze bachelorproef richt zich op het ontwikkelen van een mobiele applicatie die personal trainers en fitnesscoaches ondersteunt bij het geven van nauwkeurige feedback op de uitvoering van krachttrainingsoefeningen. 
Door het toenemende aantal mensen dat sport en traint op afstand, ontstaat er een groeiende behoefte aan digitale hulpmiddelen die bewegingen kunnen analyseren en vergelijken met referentiemateriaal om zo blessures door foutieve uitvoering te voorkomen.

De centrale onderzoeksvraag luidt: hoe kan een applicatie ontwikkeld worden die de bewegingspatronen van cliënten effectief analyseert en vergelijkt met referentievideo’s om bruikbare feedback te geven? Om deze vraag te beantwoorden is het onderzoek opgedeeld in meerdere fasen. 
Eerst is een literatuurstudie uitgevoerd om de meest geschikte pose estimation-technologieën en relevante parameters voor bewegingsvergelijking te identificeren. 
Vervolgens werd een dataset samengesteld met videobeelden van correcte en incorrecte oefeningsuitvoeringen. 
Op basis van deze data is een analysekader ontwikkeld dat automatisch lichaamshoeken berekent en vergelijkt met behulp van algoritmen.

Het resultaat is een proof-of-concept mobiele applicatie die videobeelden van krachttrainingsoefeningen analyseert, vergelijkt met referentiemateriaal en realtime feedback geeft over de technische uitvoering. 
Evaluatie toonde aan dat het systeem in staat is om bewegingsverschillen te detecteren en bruikbare feedback te genereren, wat bijdraagt aan het verbeteren van trainingsresultaten en het voorkomen van blessures.

