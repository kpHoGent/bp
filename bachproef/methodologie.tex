%%=============================================================================
%% Methodologie
%%=============================================================================

\chapter{\IfLanguageName{dutch}{Methodologie}{Methodology}}%
\label{ch:methodologie}

%% TODO: In dit hoofstuk geef je een korte toelichting over hoe je te werk bent
%% gegaan. Verdeel je onderzoek in grote fasen, en licht in elke fase toe wat
%% de doelstelling was, welke deliverables daar uit gekomen zijn, en welke
%% onderzoeksmethoden je daarbij toegepast hebt. Verantwoord waarom je
%% op deze manier te werk gegaan bent.
%% 
%% Voorbeelden van zulke fasen zijn: literatuurstudie, opstellen van een
%% requirements-analyse, opstellen long-list (bij vergelijkende studie),
%% selectie van geschikte tools (bij vergelijkende studie, "short-list"),
%% opzetten testopstelling/PoC, uitvoeren testen en verzamelen
%% van resultaten, analyse van resultaten, ...
%%
%% !!!!! LET OP !!!!!
%%
%% Het is uitdrukkelijk NIET de bedoeling dat je het grootste deel van de corpus
%% van je bachelorproef in dit hoofstuk verwerkt! Dit hoofdstuk is eerder een
%% kort overzicht van je plan van aanpak.
%%
%% Maak voor elke fase (behalve het literatuuronderzoek) een NIEUW HOOFDSTUK aan
%% en geef het een gepaste titel.

In dit hoofdstuk wordt de methodologie van het onderzoek toegelicht. Het onderzoek is opgedeeld in verschillende fasen, elk met een specifieke doelstelling, deliverables en onderzoeksmethoden. Hieronder worden deze fasen kort beschreven, gevolgd door een verantwoording van de gekozen aanpak.

\section{Fase 1: Literatuurstudie}
\subsection{Doelstelling}
Het doel van deze fase is om een grondig overzicht te krijgen van de bestaande technologieën en methoden voor pose estimation en bewegingsanalyse, evenals de meest voorkomende krachttrainingsoefeningen.  

\subsection{Deliverables}
\begin{itemize}
    \item Een overzicht van de meest geschikte pose estimation-frameworks.
    \item Een lijst van relevante parameters voor bewegingsvergelijking.
    \item Een selectie van de meest voorkomende krachttrainingsoefeningen.
\end{itemize}

\subsection{Onderzoeksmethode}
Er wordt een uitgebreide literatuurstudie uitgevoerd, waarbij wetenschappelijke artikelen, technische documentatie en case studies worden geanalyseerd. De focus ligt op het identificeren van de meest geschikte technologieën en het bepalen van de relevante parameters voor bewegingsanalyse.  

\subsection{Verantwoording}
Een literatuurstudie is essentieel om een solide theoretische basis te leggen voor het onderzoek. Het biedt inzicht in de huidige stand van de techniek en helpt bij het maken van gefundeerde keuzes voor de volgende fasen.  

\section{Fase 2: Dataverzameling}
\subsection{Doelstelling}
Het doel van deze fase is het verzamelen van videobeelden van de geselecteerde krachttrainingsoefeningen (zie Fase 1), zowel correcte als incorrecte uitvoeringen. 
Deze beelden vormen de basis voor het analyseren en vergelijken van bewegingspatronen in latere fasen.

\subsection{Deliverables}
\begin{itemize}
    \item Een eigen samengestelde dataset met gelabelde videobeelden van correcte en incorrecte uitvoering van krachttrainingsoefeningen.
\end{itemize}

\subsection{Onderzoeksmethode}
Er werd zelf beeldmateriaal opgenomen. Deze opnames zijn systematisch gelabeld en geannoteerd om de variaties in uitvoering correct in kaart te brengen.

\subsection{Verantwoording}
Het zelf opnemen van video’s garandeert controle over camerastandpunten, herhalingen en techniekvariaties. 
Hierdoor kan gericht beeldmateriaal worden verzameld dat aansluit bij de doelstellingen van het onderzoek en kunnen specifieke scenario’s of fouten bewust worden vastgelegd.

\section{Fase 3: Analyse en algoritmische verwerking}
\subsection{Doelstelling}
Het doel van deze fase is het ontwikkelen van een analysekader dat videobeelden automatisch vergelijkt met referentie-uitvoeringen, op basis van pose estimation.

\subsection{Deliverables}
\begin{itemize}
    \item Een werkend algoritme dat gebruikersvideo’s vergelijkt met referentiemateriaal op basis van lichaamshoeken en bewegingspatronen.
    \item Inzichten in prestatieverschillen tussen correcte en incorrecte uitvoeringen.
\end{itemize}

\subsection{Onderzoeksmethode}
Het, in de eerste fase geselecteerde, pose estimation-model wordt toegepast op zowel referentie- als gebruikersvideo’s. 
Op basis van de keypoints worden hoeken tussen gewrichten berekend en vergeleken met behulp van bestaande of nieuw ontwikkelde vergelijkingsalgoritmen.

\subsection{Verantwoording}
Door gebruik te maken van een bestaand, nauwkeurig pose estimation-model wordt de nadruk verlegd van modelontwikkeling naar domeinspecifieke analyse. 
De focus ligt op het correct interpreteren van bewegingen in de context van krachttraining.  

\section{Fase 4: Proof of Concept}
\subsection{Doelstelling}
Het doel van deze fase is het ontwikkelen van een werkend prototype dat de resultaten van Fase 3 integreert in een mobiele omgeving.

\subsection{Deliverables}
\begin{itemize}
    \item Een functioneel prototype dat krachttrainingstechnieken visueel vergelijkt en feedback genereert, bruikbaar op mobiele apparaten.
\end{itemize}

\subsection{Onderzoeksmethode}
De PoC wordt gebouwd met behulp van een mobiel ontwikkelframework (zoals Flutter of React Native). 
Het pose estimation-model en de analysealgoritmen uit Fase 3 worden geïntegreerd. 
De nadruk ligt op functionele demonstratie, niet op volledige app-ontwikkeling.

\subsection{Verantwoording}
Een proof-of-concept laat toe om de toepasbaarheid van het ontwikkelde algoritme in een realistische mobiele context te testen. 
Dit vormt een waardevolle stap richting een volwaardige toepassing.

\section{Fase 5: Evaluatie en conclusie}
\subsection{Doelstelling}
Het doel van deze fase is het evalueren in hoeverre het ontwikkelde analysekader en het prototype voldoen aan de verwachtingen voor het automatisch analyseren van krachttrainingstechniek.

\subsection{Deliverables}
\begin{itemize}
    \item Een evaluatierapport met sterktes, beperkingen en aanbevelingen.
    \item Conclusies over de haalbaarheid van automatische techniekvergelijking op mobiele apparaten.
\end{itemize}

\subsection{Onderzoeksmethode}
De proof-of-concept wordt getoetst aan vooraf gedefinieerde criteria, waaronder nauwkeurigheid van feedback, consistentie van analyse en bruikbaarheid. 

\subsection{Verantwoording}
Deze evaluatie vormt de brug tussen technische implementatie en praktische toepasbaarheid. 
De bevindingen kunnen richting geven aan verdere ontwikkeling en wetenschappelijk onderzoek.

\section{Verantwoording van de gekozen aanpak}
De gekozen methodologie is gebaseerd op een gestructureerde en iteratieve aanpak, waarbij elke fase voortbouwt op de resultaten van de vorige fase. Door te beginnen met een literatuurstudie wordt een solide theoretische basis gelegd, waarna de focus verschuift naar de praktische ontwikkeling en validatie van het model en de applicatie. Deze aanpak zorgt voor een efficiënte en effectieve uitvoering van het onderzoek, met als resultaat een bruikbare en betrouwbare mobiele applicatie.

